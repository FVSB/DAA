\documentclass{article}
\usepackage[utf8]{inputenc}
\usepackage{amsmath}
\usepackage{amsfonts}
\usepackage{graphicx}
\usepackage{multirow} % Para celdas que ocupen varias filas



\title{Problema NP-Completo}
\author{Francisco Vicente Suárez Bellón}
\date{}

\begin{document}

\maketitle
\section{Enunciado}
En el pueblo de Villazarcillo el alcalde en su plan para mejorar la 
calidad de vida de sus habitantes ha elaborado un programa donde a cada Hombre,Mujer,Perro
del pueblo serán emparejados, ningun ser que este en un emparejamiento x puede estar en un emparejamiento y $y!=x$
Para ello se crearon las siguientes reglas:
\begin{itemize}
    \item Cada Perro describia con que pareja de $(H,M)$ desea estar. (Análogo en Mujeres y Hombres).
    \item Un submatching es un conjunto de m,h,p de todos los posibles trios.
    \item Un matching es un submatching de tamaño $n$.
    \item Un (Sub)Matching es \textbf{estable} si no tiene trios bloqueantes (Un perro, mujer, hombre que se prefieren mutuamente en vez del lugar asignado actualmente).
    \item La cantidad de perros,mujeres y hombres en el programa será exactamente iguales.
\end{itemize}
Despues de realizadas las incripciones el alcalde quiere conocer si es posible tener un matching perfecto.
\section*{Análisis del problema}
El trata sobre un problema de decisión en cada triplo a escoger, osea para cada elemento del conjunto de Perros, Mujeres y Hombres 
se tiene que satisfacer que todos ellos estén en algún triplo. Además la decisión de poner un triplo en la supuesta solución puede
afectar a la elección  de otro haciendo insatisfacible si se toma esa elección y no habiendo a simple vista ninguna solución a ello.

La características del problema llevan a pensar que este es de naturaleza NP-Completo, además de su similitud con el 
problema de 3-Dimensional Matching (del cual se usará para esta demostración).

\begin{table}
    \centering
    \begin{tabular}{c|cccc}
    $\alpha_1$&$\beta_1\delta_2$&$\beta_1\delta_1$&$\beta_2\delta_2$&$\beta_2\delta_1$\\
    $\alpha_2$&$\beta_2\delta_2$&$\beta_1\delta_1$&$\beta_2\delta_1$&$\beta_1\delta_2$\\
    \hline\\
    $\beta_1$& $\alpha_2 \delta_1$ & $\alpha_1 \delta_2$ & $\alpha_1\delta_1$& $\alpha_2 \delta_2$\\
    $\beta_2$& $\alpha_2 \delta_1$ & $\alpha_1 \delta_1$ & $\alpha_2\delta_2$& $\alpha_1 \delta_2$\\
    \hline\\
    $\delta_1$ & $\alpha_1 \beta_2$ & $\alpha_1 \beta_1$ & $\alpha_2 \beta_1$ & $\alpha_2 \beta_2$\\
    $\delta_2$ & $\alpha_1 \beta_1$ & $\alpha_2 \beta_2$ & $\alpha_1 \beta_2$ & $\alpha_2 \beta_1$\\

    \end{tabular}
    \linebreak
    \begin{tabular}{|c|c|}
        \hline
        Posibles emparejamiento & Triplo desestabilizador\\
        \hline
        {$\alpha_1 \beta_1 \delta_1$ , $\alpha_2 \beta_2 \delta_2$} & $\alpha_1 \beta_1 \delta_2$\\
        {$\alpha_1 \beta_1 \delta_2$ , $\alpha_2 \beta_2 \delta_1$} & $\alpha_2 \beta_1 \delta_1$\\
        {$\alpha_1 \beta_2 \delta_1$ , $\alpha_2 \beta_1 \delta_2$} & $\alpha_1 \beta_1 \delta_2$\\
        {$\alpha_1 \beta_2 \delta_2$ , $\alpha_2 \beta_1 \delta_1$} & $\alpha_2 \beta_2 \delta_2$\\
        \hline
    \end{tabular}
    \caption{Instancia del problema para el cual no existe un emparejamiento estable}
\end{table}



\section*{Reduccion desde 3DM}

Una instancia de 3DM involucra tres conjuntos finitos de igual cardinalidad, que denotamos como $A_0$, $B_0$, y $D_0$, relacionándolos con $A$, $B$, y $D$ de 3GSM. 

Dado un conjunto de tripletas $T_0 \subseteq A_0 \times B_0 \times D_0$, el problema 3DM es decidir si existe un $M_0 \subseteq T_0$ tal que $M_0$ es una coincidencia completa, es decir, cada elemento de $A_0$, $B_0$, y $D_0$ aparece exactamente una vez en $M_0$.

Dada una instancia de 3DM $I'$, construimos una instancia correspondiente de 3GSM $I$. Aunque nuestra construcción puede adaptarse para funcionar con cualquier instancia de 3DM en general, asumiremos, para simplificar la presentación, que ningún elemento de $A_0$, $B_0$ o $D_0$ aparece en más de tres tripletas de $T_0$. Esta suposición se hace sin pérdida de generalidad.

Construimos $I$ construyendo primero un "marco" que consiste en los elementos $\alpha_1, \alpha_2 \in A$, $\beta_1, \beta_2 \in B$, y $\delta_1, \delta_2 \in D$.
Las preferencias de estos elementos no dependen de la estructura de $I'$ y se muestran en la Figura 2. En la Figura 2 y en las figuras posteriores, solo nos interesamos en los roles desempeñados por unos pocos elementos en cada lista de preferencias.
Por lo tanto, usamos la notación $\Pi_{\text{Rem}}$ para denotar cualquier permutación fija pero arbitraria de los elementos restantes.



\begin{table}[h]
    \centering
    \caption{Preferencias de los elementos $\alpha_1 \alpha_2 \beta_1 \beta_2 \delta_1 \delta_2$}
    \[
    \begin{array}{c|cccccccc}
        \alpha_1 & \beta_1\delta_1 & \beta_2\delta_1 & \beta_1\delta_2 & \dots & \Pi_{\text{Rem}} & \dots \\
        \alpha_2 & \beta_2\delta_2 & \dots & \dots &\dots & \Pi_{\text{Rem}} & \dots \\
        \dots & & & & &  \\
        \hline
        \beta_1 & \alpha_1\delta_2 & \dots & & & \Pi_{\text{Rem}} & \dots & \alpha_1\delta_1 \\
        \beta_2 & \alpha_2\delta_1 & \alpha_1\delta_1 & \dots & &  \Pi_{\text{Rem}} & \dots \\
        \dots & & & & & \\
        \hline
        \delta_1 & \alpha_1\beta_2 & \dots &  & & \Pi_{\text{Rem}} & \alpha_1\beta_1 \\
        \delta_2 & \alpha_1\beta_1&  \alpha_2\beta_2 & \dots & & \Pi_{\text{Rem}} \\
        \dots & & & & & \\
    \end{array}
    \]
    \end{table}


Demostraremos más adelante en el Lema 2 que los triples $\alpha_1 \beta_1 \delta_1$  $\alpha_2 \beta_2 \delta_2$
deben estar incluidos en cualquier matrimonio estable. Ha de notarse que $\alpha_1 \beta_1 \delta_1$ es la parte más débil en tal matrimonio
porque representa la pareja menos preferida tanto para $\beta_1$ como para $\delta_1$. Entonces, si cualquier elemento $a \in A$ esta emparejado con una pareja
que prefiere menos $\beta_1 \delta_1$ por lo tanto convierte a $a\beta_1\delta_1$ es una tripleta desestabilizadora.
\\
\\
La observación anterior nos da una estrategia que utiliza el par $\beta_1 \delta_1$ como un "Límite" en las preferencias de los elementos restantes de $A$.
Una condición necesaria para un matrimonio estable en $I$ es qe todos los elementos restantes de $A$ deben emparejarse con paraes ubicados a la izquierda del límite, es decir, $\geq \beta_1 \delta_1$. Usando información de $T_0$ para construir el conjunto de elementos que deben posicionarse 
a la izquierda del límite, aseguramos que esta condición para un matrimonio estable solo puede cumplirse si $T_0$ contiene un emparejamiento completo.
La dificultad restante es asegurar que todos los elementos de $A$ se emperejen a la izquierda del límite es suficiente para obtener un matrimonio estable. Antes de dar detalles de la construcción que proporciona la solución, primero probamos los lemas que establecen las propiedades del marco.

\subsection{Lema 1:}
Si existe un matrimonio estable $M$ para $I$ construido extendiendo el marco de la Figura 2, entonces $\alpha_1 \beta_1 \delta_1 \notin M $.

\textbf{Demostración: }Supongamos que $\alpha_1 \beta_2 \delta_1 \in M$. Dado $\alpha_1 \beta_2 \delta_1 \in M$ la pareja de $\delta_2$ no puede ser $\alpha_1 \beta_1 \lor  \alpha_2 \beta_2$.
Como se tiene $\delta_2$ entonces $\alpha_1 \beta_1$ es la única pareja en $\delta_2$ es $\delta_2 \beta_2$.
Entonces $\alpha_2 \beta_2$ es la posible pareja de $\delta_2$ en $M$. Además $\alpha_2 \beta_2$ son las primeras opciones de $\alpha_2$ y $\beta_2$ respectivamente. Por lo tanto $\alpha_2 \beta_2 \delta_2$ es un triple desestabilizador en $M$ es una contradicción.

\subsection{Lema 2:}
Si existe un matrimonio estable $M$ para $I$ construido extendiendo el marco de la Figura 2, entonces $\alpha_2 \beta_1 \delta_1 \land \alpha_2 \beta_2 \delta_2  \in M$ 

\textbf{Demostración: }Primero probamos $\alpha_1 \beta_1 \delta_1 \in M$. Supongamos que $\beta_1$ no está empearejado con $\alpha_1 \delta_1 \in M$, entonces podemos encontrar un triple desestabilizador para $M$. Existen dos posibles casos:
\subsubsection{Caso 1:}
    $\beta_1$ esta emperejado con $\alpha_1 \beta_2 in M$ implica que $\alpha_2 \beta_2 \delta_2$, $\alpha_1 \beta_1 \delta_1$ y $\alpha_1 \beta_2 \delta_1 \notin M$. Por un argumento similar al del Lema 1 $\alpha_1 \beta_2 \delta_1$ es un triple desestabilizador.
\subsubsection{Caso 2:}
$\beta_1$ no está emparejado con $\alpha_1 \delta_1$ ni con $\alpha_1 \delta_1$ $\alpha_1 \beta_2 \delta_1 \notin M$ por el Lema 1. Además $\alpha_1 \beta_1 \delta_1 \notin M$ lo que implica que $\alpha_1 \beta_1 \delta_2$ es un triple desestabilizador en este caso.
\subsubsection{Conclusiones: }
Por lo tanto, concluimos que $\alpha_1 \beta_1 \delta_1 \in M$, lo que implica que $\alpha_1 \beta_1 \delta_2 \notin M$. Ahora es fácil verificar
que si $\alpha_2 \beta_2 \delta_2 \notin M$, entonces es un triple desestabilizador.
\\
\\
Si los conjuntos de $I_0$ $(A_0,B_0,C_0)$ tienen cada uno k elementos, entonces los conjuntos de $I$ $(A,B,C)$ tienen cada uno $3k+2$ elementos. Los $\alpha_s$, $\beta_s$, $\delta_s$ que están en el marco, representan dos elementos. Los restantes $3k$ elementos se definen de la siguiente manera. 
\\
Supongamos que $A_0={a_1,a_2,\dots,a_k}$, $B_0={b_1,b_2,\dots,b_k}$ y $D_0={d_1,d_2\dots,d_k}$.
Según una suposición anterior, cada elemento $a_i \in A_0$ aparece en no más de tres triples de $T_0$. Clonamos tres copias de $a_i$ y remplazamos $a_i$ con los clones $a_1[1],a_i[2],a_i[3] $en $A$.
Las preferencias de estos clones se establecen para permitir que exactamente uno de sus emparejamientos en un matrimonio estable corresponda a un triple en $T_0$.

Para evitar que los dos clones restantes interfieran con la configuración anterior, añadimos los elementos $w_{a_i},y_{a_i}$ a $B$ y $x_a,z_a$ a $D$. En un matrimonio estable, los pares
 $w_{a_i},y_{a_i}$ y $y_a,$ deben emparejarse con dos de los clones de $a_i$dejándolos fuera de acción. Completamos los conjuntos $B$ y $D$ añadiendo a ellos los elementos de $B_0$ y $D_0$ respectivamente.
 Resumiendo $A = \{\alpha_1, \alpha_2\} \cup \bigcup_{a_i \in A_0} \{a_i[1], a_i[2], a_i[3]\}$, 
 $B = B_0 \cup \{\beta_1, \beta_2\} \cup \bigcup_{a_i \in A_0} \{w_{a_i}, y_{a_i}\}$ y $D = D_0 \cup \{\delta_1, \delta_2\} \cup \bigcup_{a_i \in A_0} \{x_{a_i}, z_{a_i}\}$.\\ 


Dado que $a_ib_{j_1}d_{l_1},a_ib_{j_2}d_{l_2}$ y $a_iB_{j_3}d_{l_3}$ son los triples que contienen $a_i$ en $T_0$
las preferencias en la Figura 3 logran los objetivos descritos anteriormente. Cuando existen menos de tres triples que contienen ai, igualamos dos o más
de las $j$ y las $l$.


\begin{table}[h!]
    \centering
   
    \begin{tabular}{c|ccccccccc}
        $\alpha_1$ & & & & & & & & & \\
        $\alpha_2$ & & & & & & & & & \\
        \vdots     & & & & & & & & & \\
        $a_i[1]$   & $w_{a_i} x_{a_i}$ & $y_{a_i} z_{a_i}$ & $b_{j_1} d_{l_1}$ & $\beta_1 \delta_1$ & $\cdots$ & $\Pi_{\text{Rem}}$ \\
        $a_i[2]$   & $w_{a_i} x_{a_i}$ & $y_{a_i} z_{a_i}$ & $b_{j_2} d_{l_2}$ & $\beta_1 \delta_1$ & $\cdots$ & $\Pi_{\text{Rem}}$ \\
        $a_i[3]$   & $w_{a_i} x_{a_i}$ & $y_{a_i} z_{a_i}$ & $b_{j_3} d_{l_3}$ & $\beta_1 \delta_1$ & $\cdots$ & $\Pi_{\text{Rem}}$ \\
        \vdots     & & & & & & & & & \\
        \hline
        $\beta_1$  & & & & & & & & & \\
        $\beta_2$  & & & & & & & & & \\
        \vdots\\
        $w_a$      & $a_i[1] x_{a_i}$ & $a_i[2] x_{a_i}$ & $a_i[3] x_{a_i}$ & $\cdots$ & & $\Pi_{\text{Rem}}$ \\
        $y_a$      & $a_i[1] z_{a_i}$ & $a_i[2] z_{a_i}$ & $a_i[3] z_{a_i}$ & $\cdots$ & & $\Pi_{\text{Rem}}$ \\
        \vdots     & & & & & & & & & \\
        $b_i$      & & $\cdots$ & & & & $\Pi_{\text{Rem}}$ \\
        \vdots\\
        \hline
        $\delta_1$ & & & & & & & & & \\
        $\delta_2$ & & & & & & & & & \\
        \vdots \\
        $x_a$      & $a_i[3] w_{a_i}$ & $a_i[2] w_{a_i}$ & $a_i[1] w_{a_i}$ & $\cdots$ & & $\Pi_{\text{Rem}}$ \\
        $z_a$      & $a_i[3] y_{a_i}$ & $a_i[2] y_{a_i}$ & $a_i[1] y_{a_i}$ & $\cdots$ & & $\Pi_{\text{Rem}}$ \\
        \vdots     & & & & & & & & & \\
        $d_i$      & & $\cdots$ & & & & $\Pi_{\text{Rem}}$ \\
        \vdots     & & & & & & & & & \\
    \end{tabular}
    \caption{Preferencias en la instancia $I$. La columna de los $\beta_1 \delta_1$ representa los límites. Las preferencias de los $\alpha$,$\beta$ y $\delta$ están mostradas en la tabla anterior. }  
\end{table}





\subsection{Lema 3:}
Este lema establece los roles de $w_{a_i}$,$x_{a_i}$,$y_{a_i}$,$z_{a_i}$.

Si existe un matrimonio $M$ estable para $I$ construido con las preferencias mostradas en la Figura 3, entonces para cada $a_i \in A_0$ existen $j_1,j_2 \in 1,2,3$; $j_1 \neq j_2$ tal que:
\begin{enumerate}
    \item $a_1[j_1]w_a,x_{a_i} \in M$
    \item $a_i[j_2]y_{a_i}z_{a_i}\in M$
\end{enumerate}

\textbf{Demostración: } Considera el triplo $a_1[j_1]w_a,x_{a_i}$ el cual representa la tercera opción preferida de $x_{a_i}$ y las primeras opciones preferidas de $a_i[1]$ y $w_{a_i}$. Se convierte en un triplo desestabilizador a menos que $x_{a_i}$
se empareje con una de sus tres primeras opciones preferidas, probando la parte $1$ del lema.\\
\\
De manera similar $z_{a_i}$ debe emparejarse con una de sus tres primeras opciones preferidas. De lo contrario,$y_{a_i}z_{a_i}$ forma un triplo desestabilizador con $a_i[1] \lor a_i[2]$ dependiendo de cuál de los clones de $a_i$ esté emparejado en la parte $1$.

\textbf{Ahora estamos listos para probar la NP-completitud de 3GSM mostrando que $I$ tiene un matrimonio estable si y solo si $T_0$ tiene un emparejamiento completo de $I_0$.}

\subsection{Teorema 1:}
Si $T_0$ contiene un emparejamiento completo $M_0$ de la instancia 3DM $I_0$, entonces la instancia 3GSM construida $I$ tiene un matrimonio estable $M$.
\\

Mostramos que es posible construir un matrimonio estable $ M $. Comenzamos agregando $\alpha_1 \beta_1 \delta_1$ y $\alpha_2 \beta_2 \delta_2$ a $M$.

Por cada elemento de $a_i \in A_0$, los únicos triplos en $T_0 $ que contienen $a_i$ son $a_i b_{j_1} d_{l_1} $, $a_i b_{j_2} d_{l_2} $
y $a_i b_{j_3} d_{l_3} $ usando las notaciones descritas en la tabla anterior. Uno de esos triplos esta en $M_0$.

Añadir a M si:
\begin{itemize}
    \item $a_i[1]b_{j_1}d_{l_1}$, $a_1[2]w_{a_i}x_{a_i}$, $a_i[3]y_{a_i}z_{a_i}$ If: $a_i b_{j_1} d_{l_1} \in M_0$ 
    \item $a_i[1]w_{j_i}x_{a_i}$, $a_1[2]b_{j_2}d_{l_2}$, $a_i[3]y_{a_i}z_{a_i}$ If: $a_i b_{j_2} d_{l_2} \in M_0$ 
    \item $a_i[1]w_{a_i}x_{a_i}$, $a_1[2]y_{a_i}z_{a_i}$, $a_i[3]b_{j_3}d_{l_3}$ If: $a_i b_{j_3} d_{l_3} \in M_0$ 
\end{itemize}

Dado que $M_0$ es un emparejamiento completo, la construcción anterior garantiza que los elementos de $B$ y $D$ que se originan de $B_0$ y $D_0$ se usen exactamente una vez en M. 
Es fácil verificar que todos los demás elementos de $A$, $B$ y $D$ también se usan exactamente una vez. Por lo tanto,$M$ es un matrimonio.


Para demostrar que $M$ es estable, es suficiente mostrar que ningún elemento de $A$ es un componente de un triplo desestabilizador. $\alpha_1$ y $\alpha_2$ satisfacen esta condición de inmediato porque
están emparejados con sus primeras opciones de preferencia.

Refiriéndose a la Figura 3, cada uno de los elementos restantes de $A$
está emparejado con un par ubicado a la izquierda del límite. Por lo tanto, los únicos pares
que pueden formar triplos desestabilizadores son $w_{a_i} x_{a_i}$ y $y_{a_i}z_{a_i}$. Sin embargo,
el emparejamiento de los $w_{a_i}$ ($y_{a_i}$) es una de sus tres primeras
opciones de preferencia.
Estas tres opciones están en orden inverso
exacto a las de $x_{a_i}$ ($z_{a_i}$). Esto elimina a $w_{a_i}$ y $y_{a_i}$ de participar en cualquier
triplo desestabilizador.

\subsection{Teorema 2:}
Si la instancia 3GSM $I$ tiene un matrimonio estable,
entonces $T_0$ contiene un emparejamiento completo de la instancia 3DM $I_0$.
\textbf{Demostración:}
Supongamos que $I$ tiene un matrimonio estable $M$. El Lema 2 requiere que $M$ incluya a 
$\alpha_1 \beta_1 \delta_1$  y $\alpha_2 \beta_2 \delta_2$ . El Lema 3 requiere que, para cada $a_i \in A_0$,
dos de los clones de $a_i$ se emparejen con $w_{a_i} x_{a_i}$ y $y_{a_i} z_{a_i}$.
Sea $M_0$ el emparejamiento que resulta cuando $M$ se restringe a los elementos
restantes que no tienen emparejamientos predeterminados.

Para cada ai en $a_i \in A_0$, solo queda un clon de $a_i$ por emparejar en $M_0$.
Por lo tanto, eliminaremos la distinción entre un clon de $a_i$ y el $a_i$ que representa,
sin el riesgo de introducir ninguna ambigüedad en $M_0$.
Los elementos que participan en $M_0$ pueden caracterizarse como
exactamente aquellos elementos de $A_0$, $B_0$ y $D_0$.
Dado que $M_0$ es un subconjunto de un matrimonio,representa un emparejamiento completo.


Debido a la ausencia de tríos desestabilizadores,
cada $a_i$ en $M_0$ debe emparejarse con una opción 
de preferencia ubicada a la izquierda del límite. 
La construcción de $I$, como se ilustra en la Figura 3, 
restringe esta opción al tercer elemento de la lista de preferencias, 
ya que los dos primeros elementos ya están emparejados. 
Además, el triplo formado por $a_i$ y este elemento está contenido en
$T_0$. 
Por lo tanto, cada trío en $M_0$ también es un trío en $T_0$, y $M_0$
es el emparejamiento completo deseado contenido en $T_0$.

\subsection{Conclusión de la demostración:}
Es fácil verificar que la construcción de
$I$ a partir de $I_0$ se puede llevar a cabo
dentro de un límite de tiempo polinómico.
Por lo tanto, los Teoremas 1 y 2 establecen que
3GSM es NP-duro.
También es posible verificar la estabilidad de un
matrimonio dado en tiempo polinómico, 
estableciendo así la pertenencia de 3GSM a NP.

Por lo tanto el problema es NP-completo


\end{document}


