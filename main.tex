\documentclass{article}
\usepackage{graphicx} % Required for inserting images

\title{DAA}
\author{Francisco Vicente Suárez Bellón}
\date{September 2024}

\begin{document}

\maketitle

\section{Ejericios Proyecto Final DAA}

\subsection{Ejercicio 1}
\title{1990 D. Grid Puzzle }
You are given an array a
 of size n
.

There is an n×n
 grid. In the i
-th row, the first ai
 cells are black and the other cells are white. In other words, note (i,j)
 as the cell in the i
-th row and j
-th column, cells (i,1),(i,2),…,(i,ai)
 are black, and cells (i,ai+1),…,(i,n)
 are white.

You can do the following operations any number of times in any order:

Dye a 2×2
 subgrid white;
Dye a whole row white. Note you can not dye a whole column white.
Find the minimum number of operations to dye all cells white.

Input
The first line contains an integer t
 (1≤t≤104
) — the number of test cases.

For each test case:

The first line contains an integer n
 (1≤n≤2⋅105
) — the size of the array a
.
The second line contains n
 integers a1,a2,…,an
 (0≤ai≤n
).
It's guaranteed that the sum of n
 over all test cases will not exceed 2⋅105
.

Output
For each test case, output a single integer — the minimum number of operations to dye all cells white

\end{document}
