\documentclass{article}
\usepackage{amsmath}
\usepackage{graphicx}
\usepackage{longtable}
\usepackage{amsfonts}

\title{Ejercicios de Programación Dinámica}
\author{Francisco Vicente Suárez Bellón}
\date{Septiembre de 2024}

\begin{document}

\maketitle

\section*{Problema}
El problema trata explícitamente de qué subsecuencia puedo tomar tal que maximice el valor. Si se pudieran generar todas las subsecuencias posibles, entonces se pudiera encontrar cuáles son las óptimas.
\begin{itemize}
    \item En nuestro caso, cualquier subsecuencia óptima tiene el mismo valor, por lo que todo elemento del conjunto óptimo global es óptimo igual.
    \item Pueden existir múltiples óptimos:
\end{itemize}

Para $a=5$ y $b=1$:

\[
\begin{array}{|c|c|c|c|c|}
\hline
v & 1 & 1 & 1 & 1 \\
\hline
c & 1 & 2 & 1 & 2 \\
\hline
index & 0 & 1 & 2 & 3 \\
\hline
\end{array}
\]

El máximo puede ser con 0-2-3 o 0-1-3, que es 7.

\section*{Observaciones}
\begin{itemize}
    \item Si tenemos una bola que, al multiplicar con $a$ o $b$, da un valor $< 0 \Rightarrow$ esa bola (si es que la última antes de añadir es del mismo color o no) no puede ser añadida.
    \item El valor máximo global requiere que todos los factores de las bolas sean positivos.
    \item Un valor de cero en una bola no afecta si está al final de la cadena máxima, pero puede influir si está en el medio, ya que la siguiente bola puede ser de color análogo, haciendo que esta esté en la cadena máxima.
\end{itemize}

\section*{Solución Fuerza Bruta}
Si generamos todas las subcadenas posibles, entonces podemos saber cuáles nos dan el máximo y tomar este valor. El algoritmo sería similar a generar todas las cadenas binarias, ya que es análogo a poner esa bola como última en la secuencia.
\begin{itemize}
    \item Esta solución es exponencial, por lo tanto, no es eficiente ni cerca de serlo.
\end{itemize}

\subsection*{Podas}
\begin{itemize}
    \item Se puede pensar en no añadir una bola dado que el valor multiplicado por $a$ o $b$ lo que corresponda es $< 0$, lo que descartaría casos que no llevan al óptimo.
    \item Sin embargo, en ciertos casos es necesario, por ejemplo:
\end{itemize}

\[
\begin{array}{|c|c|c|c|c|c|c|}
\hline
v & 1 & -2 & 1 & 4 & 0 & -1 \\
\hline
c & 1 & 2 & 1 & 2 & 1 & 1 \\
\hline
index & 0 & 1 & 2 & 3 & 4 & 5 \\
\hline
\end{array}
\]

Para $a=2$ y $b=1$, escoger los índices 0-1-8 es conveniente.
\begin{itemize}
    \item Esto se debe a que el valor máximo si la subsecuencia termina con el color $c$ es $-\infty$, dado que nunca se ha escogido ese color.
    \item Si ese color ya tuvo representación en alguna subcadena explorada, poner un negativo después de multiplicar con $a$ o $b$ no puede llevar a una solución óptima.
\end{itemize}

\section*{Ideas para Optimizar la Solución}
\begin{itemize}
    \item Si tienes una bola de color $c[i]$, que con alguna antecesora de color $c[i-1]$ mejora el valor de la secuencia, puedes terminar la secuencia con ese color.
    \item Si una bola tiene valor $=0$, puede afectar el valor óptimo dependiendo de si se añade o no, y su valor puede influir en las siguientes bolas.
\end{itemize}

No importa en qué momento estemos, siempre podemos tener, para cada color, el conjunto de subsecuencias máximas hasta el momento. 

\section*{Algoritmos Propuestos para Optimizar en Tiempo Polinomial}
Dado que llevamos las subsecuencias óptimas (una por cada color) hasta la bola $i$, podemos para cada color conocer si es conveniente añadir esa bola con la última bola de esa subsecuencia.

\subsection*{Algoritmo en $O(n \cdot c)$}
\subsubsection*{Idea}
Supongamos que tenemos, por cada color, el valor máximo de todas las cadenas hasta el momento que terminan con ese color. Si recorremos el array de bolas, por cada bola determinamos si el valor máximo de nuestro color es menor que unirnos con alguna subsecuencia que termine en otro color.

\subsubsection*{Demostración}
Cada vez que seleccionamos una bola, miramos cuál es la bola que debe precederla. Comparando todos los colores, si el valor de alguno es mayor que el de nuestro propio color, actualizamos.

\subsection*{Algoritmo en $O(n)$}
\subsubsection*{Notación}
\begin{itemize}
    \item $Max_1$: Valor de las cadenas óptimas hasta el momento.
    \item $Max_2$: Segundo mejor valor de las cadenas óptimas hasta el momento.
    \item $v[i]$: Valor de la bola $i$.
\end{itemize}

\subsubsection*{Idea}
En cada iteración nos interesa conocer el valor de las subcadenas óptimas que terminan en diferentes colores. Si dos colores tienen el mismo valor óptimo hasta el momento, los tratamos como equivalentes, y guardamos el valor máximo de todas las subcadenas óptimas en la variable $Max_1$, y el segundo mejor valor en $Max_2$.
\end{document}
