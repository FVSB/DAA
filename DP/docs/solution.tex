\documentclass{article}
\usepackage{amsmath}
\usepackage{graphicx}
\usepackage{longtable}
\usepackage{amsfonts}
\usepackage{listings}
\usepackage{float}

\title{Ejercicios de Programación Dinámica}
\author{Francisco Vicente Suárez Bellón}
\date{Septiembre de 2024}

\begin{document}

\maketitle
\pagebreak
\section*{Definición formal del problema}
Hay $n$ bolas. Están dispuestas en una fila. Cada bola tiene un color (para conveniencia, un número entero) y un valor entero. El color de la $i$-ésima
bola es $c_i$ y el valor de la $i$-ésima bola es $v_i$.\\
La ardilla Liss elige algunas bolas y forma una nueva secuencia sin cambiar el orden relativo de las bolas. Ella quiere maximizar el valor de esta secuencia.
El valor de la secuencia se define como la suma de los siguientes valores para cada bola (donde *a* y *b* son constantes dadas): 
\begin{enumerate}
    \item Si la bola no está al principio de la secuencia y el color de la bola es el mismo que el de la bola anterior, suma (el valor de la bola ) x $a$ .
    \item De lo contrario, suma(el valor de la bola) x $b$.
\end{enumerate}


Se te dan $q$ consultas. Cada consulta contiene dos enteros $a_i$ y $b_i$ .

Para cada consulta, encuentre el valor máximo de la secuencia que puede crear cuando $a=a_i$ y $b=b_i$ .

\textbf{Nota:} Ten en cuenta que la nueva secuencia puede estar vacía, y el valor de una secuencia vacía se define como cero.


\pagebreak

\section*{Solución Fuerza Bruta}
El problema puede tratarse como en cada bola tomar la decisión de añadirla a la subsecuencia. Si se generan todas las subsecuencias posibles entonces
se puede seleccionar la de valor máximo.

Puede ser visto como generar todas las cadenas binarias donde en caso de la bola en la posicion i sea $True$ entonces se añadirá a la subsecuencia y en caso contrario no se añadirá
aquí seleccionar el conjunto de estas cuyo valor sea máximo (Cualquier subsecuencia de este conjunto a efectos del problema son iguales dado que tiene igual valor).

\begin{itemize}
    \item Esta solución es exponencial $O(2^n)$
\end{itemize}

\subsection*{Podas}
\begin{itemize}
    \item Se puede pensar en no añadir una bola dado que el valor multiplicado por $a$ o $b$ lo que corresponda es $< 0$, lo que descartaría casos que no llevan al óptimo.
    \item Sin embargo, en ciertos casos es necesario, por ejemplo:
\end{itemize}

\[
\begin{array}{|c|c|c|c|c|c|c|}
\hline
v & 1 & -2 & 1 & 4 & 0 & -1 \\
\hline
c & 1 & 2 & 1 & 2 & 1 & 1 \\
\hline
index & 0 & 1 & 2 & 3 & 4 & 5 \\
\hline
\end{array}
\]

Para $a=2$ y $b=1$, escoger los índices 0-1-8 es conveniente.
\begin{itemize}
    \item Esto se debe a que el valor máximo si la subsecuencia termina con el color $c$ es $-\infty$, dado que nunca se ha escogido ese color.
    \item Si ese color ya tuvo representación en alguna subcadena explorada, poner un negativo después de multiplicar con $a$ o $b$ no puede llevar a una solución óptima.
\end{itemize}
\begin{figure}[H]
    \centering

%Aquí hay un ejemplo de código: \lstinline|print("Hola, mundo")|.
\begin{lstlisting}[language=Python]
    def solver(n,q,v,c):
    resultado = generar_combinaciones(n)

    for _ in range(q):
        a, b = map(int,input().split())
        chains:list[Manager]=[]
        for combinacion in resultado:
            manager=Manager(a=a,b=b)
            for i in range(n):
                if not combinacion[i]:
                    continue
                manager.add(index=i,value=v[i],color=c[i])
            chains.append(manager)
        chains=sorted(chains,reverse=True)
        #print(chains)
        print(f"El mayor valor es {chains[0].value}")
    \end{lstlisting}
\end{figure}
\subsection{Correctitud del algoritmo}
Como tenemos un algoritmo que genera todas las cadenas binarias de longitud $n$ entonces solo tenemos que recorrer esa combinación
y tener esa subsecuencia, después ordenamos desde mayor a menor por lo cual tenemos las subsecuencias de mayor valor
\subsection{Complejidad}
\subsubsection{Temporal}
\begin{itemize}
    \item Obtener todas las cadenas binarias tiene un costo de $O(2^n)$
\end{itemize}
\subsubsection{Espacial}
\begin{itemize}
    \item Tener todas las cadenas binarias tiene un coste de $O(n*(2^n))$ 
\end{itemize}

\pagebreak %% Pasar a reformular el problema

\section*{Reformular el problema}

\begin{itemize}
    \item En nuestro caso, cualquier subsecuencia óptima tiene el mismo valor, por lo que todo elemento del conjunto óptimo global es óptimo igual.
    \item Pueden existir múltiples óptimos:
\end{itemize}

Para $a=5$ y $b=1$:

\[
\begin{array}{|c|c|c|c|c|}
\hline
v & 1 & 1 & 1 & 1 \\
\hline
c & 1 & 2 & 1 & 2 \\
\hline
index & 0 & 1 & 2 & 3 \\
\hline
\end{array}
\]

El máximo puede ser con 0-2-3 o 0-1-3, que es 7.

\section*{Observaciones}
\begin{itemize}
    \item Si tenemos una bola que, al multiplicar con $a$ o $b$, da un valor $< 0 \Rightarrow$ esa bola (si es que la última antes de añadir es del mismo color o no) no puede ser añadida.
    \item El valor máximo global requiere que todos los factores de las bolas sean positivos.
    \item Un valor de cero en una bola no afecta si está al final de la cadena máxima, pero puede influir si está en el medio, ya que la siguiente bola puede ser de color análogo, haciendo que esta esté en la cadena máxima.
\end{itemize}


\section*{Notación}
\begin{itemize} 
  \item $W[j]$ Conjunto de subcadenas que terminan en j.
  \item $W^*[j]$ Conjunto de subcadenas que terminan en j con valor máximo.
  \item $w^*_j$ subcadena representante del conjunto $W^*[j]$.
  \item $Val[w]$ valor de la cadena $w$
  \item $Col[i]$ color de la bola $i$
  \item $Col[w]$ color de la última bola de la cadena w.
  \item $W^*_\alpha[i]$ Conjunto de cadenas de valor máximo de color $\alpha$ hasta la bola $i$
  \item $w^*_\alpha[i]$ subcadena que es representante del conjunto $W^*_\alpha[i]$ 
  \item $End[i]$ Conjunto de subcadenas que donde terminan con la $bola[i]$
  \item $End^*[i]$ Conjunto de  subcadenas de valor máximo que terminan con la  $bola[i]$
  
\end{itemize}

\section*{Optimizar la Solución}
 

Podemos ver el problema como un problema de decisión en el cual que subcadena $W$ (puede ser la subcadena vacía) unimos la bola $i$
por tanto para ello tenemos que ver por cada bola $j$ $(j<i)$ cual es la subcadena con la cual podemos unir $i$ y que 
maximize el valor de esta.
Inicialmente vemos que para toda $w^*_j \in W^*[j] $, $w^`=w^*_j + bola[i]$ el $Val[w^`]$ por tanto cualquier $w^*_j$
añadir $bola[i]$ será el mismo valor

\begin{table}
    \centering
    \begin{tabular}{|c|c|}
        \hline
        \dots \dots & $bola_i$\\
        \hline
    \end{tabular}
    \caption{Ejemplo de como se debe unir la bola}
\end{table}

\subsection{Lema 1: }
Para todo $w^*_j \in W^*[j] $ $w^`=w^*_j + bola[i]$  el $Val[w^`]$ sera el mismo.

\subsection{Nota}
Dado lo anterior es evidente que una vez computado el $W^*[j]$ no es necesario recomputarlo en el resto de los llamados.
Por lo tanto estamos ante una subestrutura óptimam, de las cuales una vez calculado se puede guardar en una caché
de aquí partirá el resto de nuestras Ideas


\section{Algoritmo en $O(n * c )$}
Dado que anteriormente vimos que existe una subestructura óptima con respecto a guardar en cada bola cual es la mejor 
subcadena en la cual ella es la última bola.
Partiendo de esta idea, en el problema no tiene importancia en si cual es la bola anterior, solo nos importa cual es la cadena $w^*_\alpha[i]$
por cada color. 


\subsection{Lema 2:}
Para todas subcadenas hasta $i$ con igual color la que se elegirá para unir a $bola[i]$ es la de mayor valor.
Sean $h,q$ bolas; $h,q <i$ y $Col[h] = Col[q]$
Si $Val[w^*_h] = Val[w^*_q]$ entonces al momento de elegir en la bola[i] cualquiera de las dos cadenas son iguales
Si $Val[w^*_h] < Val[w^*_q]$ entonces siempre elegiremos a la $w^*_q$ dado que al poner a la $bola[i]$ al final de estas
la de mayor valor será óptima para las cadenas que terminan en el Col[h].
\subsubsection{Algoritmo}
Ahora conociendo que para cada color $\alpha$ la subcadena que se seleccionará para unir a la $bola[i]$.
Osea  $w^*_\alpha[i] \in End[i]$ por lo tanto $End^*[i]$ están las subcadenas de los colores a los cuales añadir la $bola[i]$ hacen que su 
valor sea óptimo para esta.

Dado lo anterior podemos elaborar un algoritmo de programación dinámica donde iteraremos por cada bola , en ella iteramos por cada color, 
para ir actualizando el valor de la cadena máxima contando con la $bola[i]$ con una chaché $dp[\alpha]$ donde guarda el valor de la cadena de valor máximo
en el color $\alpha$ de la siguiente forma:

\begin{itemize}
  \item $ dp[Col[bola[i]]]=max(dp[Col[bola[i]]],dp[\beta]+(b*v[i]), dp[Col[bola[i]]]+a(a*v[i]), b*v[i])$ para todo color $\beta \neq Col[bola[i]]$
\end{itemize}

\subsection{Subestructura Óptima}
Como se ha visto la función de asignacion de $w^*_\alpha[i]$ depende de los valores ya precomputados de las bolas que se han visto con anterioridad.
Por tanto supongamos que ya computado la subcadena de valor máximo del color de la $bola[i]$ hasta la $bola{i}$ (Ya se computo esta)
es $s^*$ será el máximo entre lo ya computado entre todos los colores (también puede ser que lo subóptimo sea empezar en esta bola para este color en específico )
Por tanto si obtengo una solución optima global del problema $>0$ es porque la subcadena es distinta del vacío.

\subsection{Correctitud de la DP}
Para demostrar que la solución por DP funciona vamos a realizar una inducción sobre la cantidad de elementos
\textbf{Caso Base : }
\begin{itemize}
    \item Para $n=1$ Se reduce a $max(b*v[i],0)$
    \item Para $n=2$ Vemos que se cumple la propiedad dado que la primera bola y su color respectivo se basan en caso base anterior 
     por lo tanto el color que representa la primera bola tiene el subóptimo hasta dicha bola, por tanto la 2da bola dada la propia función solo puede mejorar 
     o igualar la solución de los colores precomputados

\subsubsection{Hipótesis :}
Supongamos que para todo array con k bolas se cumple la propiedad de dp
\subsubsection{Tesis:}
Supongamos que tenemos un array de tamaño k+1 bolas, y lo separamos en 2 secciones, $x,y$ donde $x$ tendrá k elementos y $y$ un elemento
por hipótesis x tiene precomputado correctamente el dp.
Hasta x se tienen precomputado correctamente por cada color cual es su subsecuencia máxima, como anteriormente se ha demostrado que al añadir una
bola para maximizar la subcadena máxima es suficiente con el valor de las subcadenas óptimas para cada color precomputadas hasta ese momento.
por tanto para unir a $y$ buscamos los valores de dp por cada color (incluido el vacío) y computamos la función de matcheo y elegimos la de mayor valor.

     
\end{itemize}


\begin{figure}[H]
    \centering
    \begin{lstlisting}[language=Python]
        for _ in range(q):
            
            dp=[-INF]*(N)
            color_can=[False]*len(dp) # Llevamos la 

            for i in range(n): # Por cada elemento
                ci=c[i]
                last=dp[ci]

                for col in range(1,len(dp)):

                    if col == ci and color_can[ci]:

                        s=last if last>-INF else 0
                        dp[ci]=max(dp[ci],s+(a*v[i])) 
                    else:
                        s=dp[col] if dp[col]>-INF else 0
                        dp[ci]=max(dp[ci],s+(b*v[i]))

                dp[ci]=max(dp[ci],b*v[i])
                color_can[ci]=True

            print(max(max(dp),0))
    \end{lstlisting}
\end{figure}
 
\subsection{Demostración de la correctitud del algoritmo}
En este algoritmo por cada bola se recorre cada uno de los colores actualizando la dp, es evidente que para la primera
ocurrencia de un color se rellena por primera vez el valor de cada color con el factor $b*v[i]$.
Para cada bola hay 3 opciones.
\begin{enumerate}
    \item Que exista una subcadena que termine en mi mismo color al cual al yo ponerme al final de este mejores la subcadena.
    \item Que añadirme en el final de otra subcadena un color distinto al mio mejore con respecto al subóptimo que tenía
    \item Que comenzar la subcadena pueda ser mas ventajoso
\end{enumerate}


\subsection{Complejidad}
\subsubsection{Temporal:}
Como se itera por cada bola $O(n)$ y por cada una se itera por todos los colores $O(C)$ y todas las operaciones dentro
de este bucle son constantes en tiempo. El algoritmo es $O(c*n)$.

\subsubsection{Espacial:}
Dado que la creación de las caches se suguiere hacerse con arreglos (los cuales permiten indexar en $O(1)$) 
El coste de estos son $O(C)$ siendo $C$ la cantidad de colores distintos.

\section*{Algoritmo en $O(n)$}
Basandonos en las mismas ideas demostradas en el anterior algoritmo de programación dinámica. 
Para poder eliminar cómputo innecesario podemos fijarnos en el caso que la bola a añadir se quiere poner al
final de una subcadena que termina en un color distinto a este por lo tanto el valor de crear esa nueva subcadena es de
$\text{valor subcadena color distinto}+(b*\text(valor de esa bola))$ por lo tanto intuitivamente si maximizamos el valor de la 
subcadena a la cual nos unimos entonces será la mejor entre todas las posibles.

Entonces si llevamos la cuenta de cual es el valor de las subcadenas máximas hasta cada bola, solo nos quería tocar el caso
que la bola tenga un color igual al del $max$, por tanto debemos llevar el $max_1$ (máximo total) y $max_2 $(segundo mayor máximo)
Dado que se compara en este caso a añadir esa bola con una subcadena con valor  $max_2 $ (Aquí también se tiene en cuenta el empezar la cadena desde esa bola).
\\
\textbf{Nota:} $max_1 $ y $max_2 $ son $\geq 0$

\subsection{Lema 3:}
Demostrar que dado una$ bola[i]$ de color $Col[bola[i]]$ de todas las subcadenas que se pueden formar tal que termine con esta bola
y la penúltima bola tenga un color distinto a la $bola[i]$  la de mayor valor, $s^*$ es la $s^*-bola[i]$ con mayor valor.

Supongamos que tenemos $s^*$ que esta formada por $s_1+i$, $s_1$, $Col[w_1] \neq Col[bola[i]]$ no es la de mayor
valor entre las que tienen distinto color. 
Sea $s_2$, $Col[s_2] \neq Col[bola[i]]$ y $s_2$ tiene valor máximo entre las subcadenas que no terminan en  $Col[bola[i]]$

Entonces si $Val[s^*]= Val[s_1]+(b*v[bola[i]])$ y $Val[s_m]= Val[s_2]+(b*v[bola[i]])$ , $Val[s_1]<Val[s_2]$ Contradiccion.
$s_m$ máxima.

Dado que ahora no es necesario computar por las cadenas terminadas en distinto color para conocer si 
añadir la bola mejorará la dp entonces nos hay que analizar el caso en que la mejor solución sea mi propio color
por tanto debo comprobar con las 2das mejores soluciones para verificar que elección es más conveniente.



\subsection{Subestructura Óptima}
La subestrutura óptima es análoga a la propuesta anteriormente, aunque en este caso aplicamos en Lema 3, por tanto
no hay afectación en la estructura de esta
\subsection{Correctitud de la DP}
Como en este algoritmo continuamos utilizando la misma idea dinámica anterior aplicando el Lema 3, nos queda añadir que 
no es necesario iterar por todos los colores dado que la aplicacion antes mencionada hacer que los colores distintos a la bola no 
sean necesarios. Y el caso de que el color de esa bola sea igual al del máximo se tiene el $max_2 $ para comparar si es mejor.

\subsection{Código}

\begin{figure}[H]
    \centering
    \begin{lstlisting}[language=Python]
        for _ in range(q):
    
            mx1 = 0
            mx2 = 0

            ans = [-INF] * N  # Inicializamos el arreglo ans con -INF

            for i in range(n):
                if mx1 == ans[c[i]]:
                    ans[c[i]] = max(ans[c[i]], ans[c[i]] + a * v[i])
                    ans[c[i]] = max(ans[c[i]], mx2 + b * v[i])
                    mx1 = max(mx1, ans[c[i]])
                else:
                    ans[c[i]] = max(ans[c[i]], ans[c[i]] + a * v[i])
                    ans[c[i]] = max(ans[c[i]], mx1 + b * v[i])
                    mx2 = max(mx2, ans[c[i]])

                    if mx2 > mx1:
                        mx1, mx2 = mx2, mx1

            print(mx1)
    \end{lstlisting}
\end{figure}
 
\end{document}
